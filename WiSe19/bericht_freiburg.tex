Die 81. Zusammenkunft aller deutschsprachigen Physikfachschaften (kurz: ZaPF) tagte vom 31. Oktober bis zum 3. November in Freiburg im Breisgau. Die ZaPF ist die Bundesfachschaftentagung der Physik und versteht sich dabei als eine grundlegende Basis zum Austausch zwischen den Physikfachschaften im deutschsprachigen Raum über hochschulpolitische Themen. Darüber hinaus dient sie als Gremium der Meinungsbildung und -äußerung der Physikstudierenden.

Die ZaPF tagt einmal pro Semester an unterschiedlichen Hochschulen, wobei sie von der Physikfachschaft der ausrichtenden Hochschule selbst organisiert wird. 
Im Wintersemester 2019/20 wurde die ZaPF von der Fachschaft Physik der Universität Freiburg geplant und durchgeführt.  
Es nahmen ca. 210 Fachschaftler*innen aus insgesamt 56 Fachschaften teil.
Diese tauschten sich in über 42 Arbeitskreisen aus und erarbeiteten Positionen zu verschiedenen Themen.

Bei der Tagung im Wintersemester 2019/20 waren folgende Themen Schwerpunkt: Wissenschaftskommunikation, die nationale Forschungsdateninfrastruktur (NFDI), Nachhaltigkeit, Anforderungen an Bibliotheken, die Bewegung Fridays for Future, Datenschutz und die Studienfinanzierung. 

Während der ZaPF wurden die Richtlinien für studentische Gutachter*innen in Akkreditierungsverfahren\footnote{\href{https://zapfev.de/resolutionen/wise19/akkreditierungsrichtlinien/Akkreditierungsrichtlinien.pdf}{www.zapfev.de/resolutionen/wise19/akkreditierungsrichtlinien/Akkreditierungsrichtlinien.pdf}} weiter überarbeitet und verabschiedet. Die ZaPF beschloss zudem, eine erstzeichnende Organisation des offenen Briefes zur Novellierung des Hochschulgesetzes in NRW zu sein, welcher von einem Bündnistreffen bestehend aus NRW-Fachschaften, der GEW und dem SDS verfasst wurde. Darüber hinaus wurde beschlossen, als Veranstalter des Studienreform-Forums auf der nächsten DPG Didaktik Frühjahrstagung 2020 in Bonn aufzutreten\footnote{\href{https://zapfev.de/resolutionen/wise19/studienreform/studienreform.pdf}{www.zapfev.de/resolutionen/wise19/studienreform/studienreform.pdf}}.

\section*{Wissenschaftskommunikation}
Die ZaPF hat sich über mehrere Jahre mit der Wissenschaftskommunikation auseinander gesetzt und Forderungen in der Resolution zu Wissenschaftskommunikation\footnote{\href{https://zapfev.de/resolutionen/wise19/wissenschaftskommunikation/wissenschaftskommunikation.pdf}{www.zapfev.de/resolutionen/wise19/wissenschaftskommunikation/wissenschaftskommunikation.pdf}} festgehalten. Diese Resolution wird als Ausgangspunkt genommen, um sich mit verschiedenen Akteur*innen in der Wissenschaftskommunikations-Community über die studentischen Interessen zu diesem Thema auszutauschen.\\
Im Einzelnen fordert die ZaPF eine Implementation in die Lehre und empfiehlt hierfür zum Beispiel eine fakultätsübergreifende Veranstaltung oder die Möglichkeit, seine Abschlussarbeit vor fachfremden Publikum vorzustellen, sowie die Förderung von Wissenschaftskommunikation durch die Hochschulen.

\section*{Bibliotheks- und Raumentwicklung}
Die Raum- und Bibliotheksplanung an den Universitäten erlebt - nicht zuletzt aufgrund der Digitalisierung - derzeit starke Umbrüche. Die ZaPF kritisiert den Trend, dass die Verfügbarkeit von Printmedien in Bibliotheken immer weiter reduziert und teilweise dezentrale Bibliotheken geschlossen werden. In dem Positionspapier zur Bibliotheks- und Raumentwicklung\footnote{\href{https://zapfev.de/resolutionen/wise19/bibliotheken/bibliotheken.pdf}{www.zapfev.de/resolutionen/wise19/bibliotheken/bibliotheken.pdf}} wird argumentiert, warum diese Strukturen erhalten bleiben und ausgebaut werden sollen.

Weiter spricht sich die ZaPF für mehr studentische Arbeitsräume in unterschiedlichen Formen aus. Dies fördert selbstständiges Lernen und Austausch unter Studierenden. Diese Lernräume sollten ausreichend mit Strom und Internet ausgestattet sein und möglichst rund um die Uhr zugänglich sein. Diese Forderungen sind in der Resolution zu Lern- und Arbeitsräume\footnote{\href{https://zapfev.de/resolutionen/wise19/lernraume/Lernr\%C3\%A4ume.pdf}{www.zapfev.de/resolutionen/wise19/lernraume/Lernräume.pdf}} zu finden.

\section*{NFDI}
Die ZaPF hat sich mit den aktuellen Bestrebungen der Einrichtung einer Nationalen Forschungsdateninfrastruktur (NFDI) auseinander gesetzt und befürwortet die Einrichtung dieser. In dem Positionspapier zur NFDI\footnote{\href{https://zapfev.de/resolutionen/wise19/nfdi/nfdi.pdf}{www.zapfev.de/resolutionen/wise19/nfdi/nfdi.pdf}} werden Anforderungen an eine solche Infrastruktur aus der Sicht von Physikstudierenden festgehalten.

Das Positionspapier thematisiert fünf verschiedene Aspekte, die für Studierende interessant sind: die Integration der verschiedenen Disziplinen innerhalb der Physik, die Einbindung in die Lehre, den freien Zugang, die angebotenen Dienste und deren Sicherheit sowie die Schnittstellen und Struktur.

\newpage
\section*{Fächerkombinationen im Lehramtsstudium}
Lehramtsstudierende, die andere Fächerkombinationen studieren als Mathematik und Physik, haben häufig das Problem, dass Wissen aus Mathematikveranstaltungen vorausgesetzt wird, welches sie sich zusätzlich aneigenen müssen.\\
Die ZaPF fordert deswegen, die Studierbarkeit von Fächerkombination jenseits von Mathematik und Physik sicherzustellen, indem man mehr mathematische Grundlagen in den Physikteil des Lehramtstudiums integriert. Somit setzt sich die ZaPF in dem Positionspapier zum Lehramt\footnote{\href{https://zapfev.de/resolutionen/wise19/lehramt/lehramt.pdf}{www.zapfev.de/resolutionen/wise19/lehramt/lehramt.pdf}} gegen eine Einschränkung in der Studienwahl ein.

\section*{Prüfungsunfähigkeitsbescheinigungen} 
Die ZaPF fordert, ärztliche Prüfungsunfähigkeitsbescheinigungen anzuerkennen und nicht zu verlangen, persönliche Symptome preisgeben zu  müssen.

Die Resolution zu Prüfungsunfähigkeitsbescheinigungen\footnote{\href{https://zapfev.de/resolutionen/wise19/prufungsbescheinigungen/symptompflicht.pdf}{www.zapfev.de/resolutionen/wise19/prufungsbescheinigungen/symptompflicht.pdf}} wurde als gemeinsame Resolution verschiedener Bundesfachschaftentagungen (BuFaTa) ausgearbeitet und wurde bisher durch die  Bundesfachschaftstagungen der Wirtschafts- und   Wirtschaftssozialwissenschaften  (BuFaK WiSo), der Mathematik (Konferenz der deutschsprachigen Mathematikfachschaften, KoMa),  der Psychologie  (Psychologie-Fachschaften-Konferenz, Psyfako), Elektrotechnik (BuFaTa ET), Geographie (Vertretung deutschsprachiger Geographistudierender, GeoDACH), Informatik (Konferenz der Informatikfachschaften, KIF), Materialwissenschaften (Konferenz aller werkstofftechnischen und materialwissenschaftlichen Studiengänge, KaWuM), Medizin (Bundesvertretung der Medizinstudierenden in Deutschland e.V., BVMD) und Medizintechnik (Konferenz der Medizintechnikfachschaften, KOMET) unterstützt.

\section*{Anpassung der Semesterzeiten}
In der Resolution\footnote{\href{https://zapfev.de/resolutionen/wise19/semesterzeiten/semesterzeiten.pdf}{www.zapfev.de/resolutionen/wise19/semesterzeiten/semesterzeiten.pdf}} zur Anpassung der deutschen Hochschulen an internationale Semesterzeiten schließt sich die ZaPF der \glqq Empfehlung zur Harmonisierung der Semester- und Vorlesungszeiten an Deutschen Hochschulen im Europäischen Hochschulraum\grqq{} der Hochschulrektorenkonferenz an. Dadurch soll die Studierendenmobilität gefördert werden.
Mit dieser Resolution wird die Resolution zu Internationalen Semesterzeiten bestärkt, die auf der ZaPF im Sommersemester 2016 in Konstanz verabschiedet wurde.\\
Die in Freiburg verabschiedete Resolution wurde gemeinsam von verschiedenen BuFaTas ausgearbeitet und wurde bisher durch die  Bundesfachschaftstagungen der Wirtschafts- und  Wirtschaftssozialwissenschaften  (Bufak Wiso), der Mathematik (Koma), der Psychologie  (Psyfako), Elektrotechnik (BuFaTa ET), Geographie (GeoDACH), Materialwissenschaften (KaWuM), Medizin (BVMD) und Medizintechnik (KOMET) unterstützt.

\section*{Solidarisierung mit Fridays for Future} 
Die ZaPF solidarisiert sich mit der Bewegung \glqq{}Fridays for Future\grqq{} und fordert die Umsetzung ihrer Forderungen. Insbesondere befürwortet sie das Bestreben von \glqq Fridays for Future\grqq{},  Akzeptanz für wissenschaftlich Tatsachen in der Gesellschaft zu verbreiten.\\
In der Resolution zur Solidarisierung mit Fridays for Future\footnote{\href{https://zapfev.de/resolutionen/wise19/fff/fridays_for_future.pdf}{www.zapfev.de/resolutionen/wise19/fff/fridays\_for\_future.pdf}} spricht sie sich dafür aus, dass an Schulen und Hochschulen die Freiräume geschaffen werden, dass Kinder, Jugendliche und Erwachsene an den Protesten teilnehmen können. Außerdem verurteilt sie jegliche Repressionen gegen die an den Protesten teilnehmenden Menschen. 

\section*{Open Science}
Die ZaPF hat sich wiederholt mit dem Thema Open Science beschäftigt. Die Fachschaften haben sich darüber ausgetauscht, wie verschiedene Aspekte des Themas im Studium integriert werden können. Dazu wurden Ansätze von der besseren Integration guter wissenschaftlicher Praxis ins Anfängerpraktikum bis zu Seminaren über Publikationskultur in Zusammenarbeit mit Bibliotheken diskutiert.\\
Außerdem wurde gemeinsam mit Vertreter*innen anderer Bundesfachschaftstagungen an einem interdisziplinären  Positionspapier als Erweiterung der ZaPF-Resolution zu Open Science von der ZaPF im Wintersemester 2018 in Würzburg gearbeitet.

\section*{CHE Ranking}
Verterter*innen der ZaPF sind zur Fachbeiratssitzung des CHE Hochschulrankings eingeladen, um das nächste CHE-Ranking im Fach Physik im Jahr 2021 vorzubereiten. Auf dieser ZaPF wurde sich darüber ausgetauscht, was die ZaPF im Sinne einer konstruktiven Zusammenarbeit mit dem CHE beitragen kann.
