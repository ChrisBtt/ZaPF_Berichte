Vom 19. bis 22. November 2015 fand in Frankfurt (Main) die 73. Zusammenkunft
aller deutschsprachigen Physik-Fachschaften (kurz: ZaPF) statt.  Die ZaPF ist
die Bundesfachschaftentagung der Physik und versteht sich dabei als eine
grundlegende Basis zum Austausch zwischen den Physik-Fachschaften im
deutschsprachigen Raum über hochschulpolitische Themen und darüber hinaus als
Gremium der Meinungsbildung und -äußerung der Physikstudierenden. Sie tagt
einmal pro Semester an unterschiedlichen Hochschulen, wobei sie von der
Physik-Fachschaft der ausrichtenden Hochschule selbst organisiert wird.

Diese Aufgabe wurde im Wintersemester 2015 von der Fachschaft Physik der Goethe
Universität Frankfurt übernommen. Es nahmen 194 Fachschaftler*innen aus
insgesamt 43 teilnehmenden Fachschaften teil, die sich in über 30 Arbeitskreisen
austauschten und Positionen zu verschiedenen Themen erarbeiteten.

Schwerpunkte der ZaPF in Frankfurt waren die Novellierung des
Wissenschaftszeitvertragsgesetzes (WissZeitVG), die Auseinandersetzung mit dem
CHE-Hochschulranking, die aktuelle Situation von Geflüchteten,
Wissenschaftstheorie im Physikstudium, barrierefreies Studieren, Transparenz in
der Drittmittelforschung sowie die Akkreditierung von Hochschulen und
Studiengängen.

\newpage

\section*{Novelle des Wissenschaftszeitvertragsgesetzes}
Derzeit befindet sich das Wissenschaftszeitvertragsgesetz (WissZeitVG) in der
Novellierung. Aufgrund der zahlreichen Probleme, die dieses Gesetz mit sich
gebracht hat, hat die ZaPF diese Gelegenheit genutzt, sich in einer Resolution
mit diesen zu beschäftigen und hat dabei  vorrangig die zeitlich befristete
Anstellung und die Höchstbefristungsdauern als Hauptprobleme ausgemacht.

Die zur ZaPF vorliegende Entwurfsfassung der Novelle sieht darüber hinaus eine
Maximalbefristung für studentische Hilfskräfte vor, die kürzer als die
kombinierte Regelstudienzeit von Bachelor und Master ist. Dies wäre für
zahlreiche Studierenden fatal, die auf die Einkünfte zum Bestreiten ihres
Studiums angewiesen sind.

Weiterhin löst die Novelle auch die bestehenden Probleme der Befristung für
wissenschaftliche Mitarbeiter*innen nicht und behält weiterhin die Tarifsperre bei.

Die ZaPF wendet sich mit ihrer
Stellungnahme\footnote{\href{http://zapfev.de/sites/default/files/Stellungnahme\_WiSe15\_WissZeitVG.pdf}{\url{http://zapfev.de/sites/default/files/Stellungnahme\_WiSe15\_WissZeitVG.pdf}}}
gegen diese und andere Probleme des WissZeitVG und hofft das
Gesetzgebungsverfahren positiv zu beeinflussen.

\section*{CHE-Hochschulranking}
Bereits seit 2007 beschäftigt sich die ZaPF kontinuierlich mit dem CHE-Ranking
und dessen Methoden und Veröffentlichung. Nach der Ablehnung des CHE-Rankings
2013 konnte in Zusammenarbeit mit der KFP und der jDPG 2014 wieder konstruktiv
mit dem CHE zusammengearbeitet werden und es wurde eine Arbeitsgruppe
eingerichtet. Zur Tagung dieses Semester konnten, wie auch schon im Sommer 2014,
Mitarbeiter*innen des CHE eingeladen werden. Dabei wurde insbesondere über die
momentane Berichterstattung und Veröffentlichung des CHE-Rankings geredet. Im
Zuge dessen wurde von der ZaPF ein
Positionspapier\footnote{\href{http://zapfev.de/sites/default/files/Positionspapier\_WiSe15\_CHE.pdf}{\url{http://zapfev.de/sites/default/files/Positionspapier\_WiSe15\_CHE.pdf}}}
erstellt, wie mit den Ergebnissen des Rankings in Berichterstattung umgegangen
werden soll.
%%% Thomi liest das noch mal gegen, wenn er daheim ist (gegen 5 oder morgen oder so)

\section*{Transparenz in der Drittmittelforschung}
In dem Arbeitskreis \glqq{}Tranzparenz von Drittmittelforschung\grqq{}
bearbeitet und diskutiert die ZaPF seit dem Sommersemester 2014 Fragestellungen,
wie die nach der Veröffentlichungspflicht von drittmittelfinanzierter Forschung.
Unter Berücksichtigung der Ergebnisse der vorangegangenen Arbeitskreise
vergangener ZaPFen wurde nun auf Grundlage des Resolutionsentwurfes aus Aachen
eine abschließende
Stellungnahme\footnote{\href{http://zapfev.de/sites/default/files/Stellungnahme\_WiSe15\_Transparenz\_in\_der\_Drittmittelforschung.pdf}{\url{http://zapfev.de/sites/default/files/Stellungnahme\_WiSe15\_Transparenz\_in\_der\_Drittmittelforschung.pdf}}}
entwickelt und verabschiedet.

Die ZaPF kam zu dem Schluss, mehr Transparenz zu fordern, da auch bei
drittmittelfinanzierter Forschung öffentliche Güter genutzt werden. So sollen
beispielsweise Name und Titel des Projektes bekanntgegeben werden und
Informationen zur Vertragslaufzeit und Gesamtsumme der Förderung veröffentlicht
werden.

Auf die Veröffentlichung der Ergebnisse der Forschung ist in einem weiteren AK
eingegangen worden. In diesem wurde ein Grundstein für eine Stellungnahme
gelegt, welche auf der kommenden ZaPF weiter ausgearbeitet werden soll.

\section*{Zugang zu Hochschulen für Geflüchtete}
Aufgrund der aktuellen Situation hat sich die ZaPF auch intensiv mit der
Thematik \glqq{}Geflüchtete\grqq{} auseinandergesetzt. Obwohl diese zum Teil
über eine Hochschulzugangsberechtigung oder ein bereits begonnenes Studium
verfügen, führen fehlende Unterlagen dazu, dass sie nicht am regulären
Unibetrieb teilnehmen können. Die ZaPF hat sich nun in Form einer
Stellungnahme\footnote{\href{http://zapfev.de/sites/default/files/Stellungnahem\_WiSe15\_Bildungszugang\_fuer\_Gefluechtete.pdf}{\url{http://zapfev.de/sites/default/files/Stellungnahem\_WiSe15\_Bildungszugang\_fuer\_Gefluechtete.pdf}}}
zu dieser Problematik geäußert und fordert freien Zugang zu allgemeinen
Bildungsressourcen und -infrastrukturen für Geflüchtete. Dabei haben der Bund
und die Länder dafür Sorge zu tragen, dass die genannten Maßnahmen weder den
Geflüchteten noch den Hochschulen zu Lasten fallen. Neben der
Studienfinanzierung über das BAföG muss es verstärkt unkomplizierte
Stipendienprogramme geben, um Geflüchtete gezielt zu unterstützen. Weiteres kann
in der Stellungnahme nachgelesen werden.

\section*{Wissenschaftstheorie und -ethik im Physikstudium}
Der  Arbeitskreis \glqq{}Wissenschaftstheorie im Studium\grqq{} beschäftigte
sich mit der Frage, in welcher Form und in welchem Umfang Ethik und
Wissenschaftstheorie Teil des Studiums bilden soll. Hierbei konzentrierte man
sich auf Wissenschaftsethik sowie auf korrektes wisschenschaftliches Vorgehen.
Verschiedene mögliche Inhalte wurden diskutiert und abgestimmt, ob diese
verpflichtend oder optional im Studium sein sollen. Genauere Informationen
können in dem verabschiedeten
Positionspapier\footnote{\href{http://zapfev.de/sites/default/files/Positionspapier\_WiSe15\_Wissenschaftstheorie.pdf}{\url{http://zapfev.de/sites/default/files/Positionspapier\_WiSe15\_Wissenschaftstheorie.pdf}}}
der ZaPF eingeholt werden.

\section*{Barrierefreies Studieren}
Viele Menschen mit Behinderung sind nicht über Hilfen und
Ausgleichsmöglichkeiten an ihren Universitäten informiert. Häufig werden dort
auch keine Informationen darüber weitergegeben, beispielsweise ab wann jemand
zum Erhalt eines Nachteilsausgleiches berechtigt ist. Daraus folgt leider auch,
dass Studierende sich über Thematiken wie diese nicht informieren.  Ergebnis des
Arbeitskreises war eine
Empfehlung\footnote{\href{http://zapfev.de/sites/default/files/Positionspapier\_WiSe15\_barrierefreies\_Studieren.pdf}{\url{http://zapfev.de/sites/default/files/Positionspapier\_WiSe15\_barrierefreies\_Studieren.pdf}}}
an alle Fachschaften mit Handreichung zum Umgang mit dem Themenkomplex.

\section*{Workshops zur Akkreditierung} Wie auch auf den letzten ZaPFen fanden
dieses Semester wieder Workshops zur Einführung in das deutsche
Akkreditierungswesen statt. Während auf bisherigen ZaPFen meistens Themen der
Programmakkreditierung besprochen und bearbeitet wurden, wurde dieses Semester
ein starker Fokus auf die Systemakkreditierung gelegt. Dazu waren zwei
Mitglieder des KASAP (KoordinierungsAusschuss des Studentischen
Akkreditierungs-Pools) anwesend, die den interessierten Fachschaftler*innen erst einen groben Überblick über Ideen und Konzepte der
Systemakkreditierung verschafften und dann im Detail nochmal mit ihnen an
Prozessen und Prozesssteuerung arbeiteten.

Zudem hat sich die ZaPF mit ihren Richtlinien zur Akkreditierung, die 2008
erarbeitet wurden, auseinandergesetzt. Nachdem die meisten dieser Richtlinien
von den allgemeinen Richtlinien des Akkreditierungsrats schon abgedeckt werden,
sollen die der ZaPF nun auf den kommenden Tagungen überarbeitet werden, worauf
in Frankfurt hingearbeitet wurde.

\section*{Studienführer}
Nach dem Beschluss der vorherigen ZaPF in Aachen mit der Konferenz der
Informatikfachschaften und der Konferenz der Mathematikfachschaften einen
gemeinsamen Studienführer zu  erschaffen, wurde dieses Thema weiterführend
behandelt. Neben der Aktualisierung des bestehenden
Wikis\footnote{\url{http://studienführer-physik.de}} wurden Organisatorisches
und gewünschte Features der geplanten neuen Plattform für den Studienführer
besprochen, die sowohl die Verständlichkeit für Studienanfänger*innen verbessern, als
auch den Arbeitsaufwand für teilnehmende Fachschaften verringern soll.
Darüberhinaus wurde die Erweiterung des neuen Studienführers durch andere
BuFaTas diskutiert. Geplant ist dabei  mit anderen Fachbereichen, BuFaTas und
vor allem der Bundesschülerkonferenz (BSK) als Vertretung der Zielgruppe ein
geeignetes Konzept für die Plattform zu entwickeln.

\section*{Weitere Themen}
Ergänzend zum Studienführer arbeitet die ZaPF an einer Master-Wechseldatenbank,
um Erfahrungen beim Wechsel der Hochschulen zwischen Bachelor und Master zu
dokumentieren und zusammenzuführen.

Der Arbeitskreis \glqq{}Gläserne Decke\grqq{} hat zum zweiten Mal statt
gefunden. Es wurde eine Sammlung der Frauenförderprogramme an verschiedenen
Universitäten angefangen, die bis zur nächsten ZaPF vervollständigt werden
soll.

Eine Liste aller Arbeitskreise und deren Ergebnisse werden im
Reader\footnote{\href{http://zapfev.de/sites/default/files/Reader\_ZaPF\_WiSe15\_Frankfurt.pdf}{\url{http://zapfev.de/sites/default/files/Reader\_ZaPF\_WiSe15\_Frankfurt.pdf}}}
zur ZaPF veröffentlicht.

\vfill

Die nächste ZaPF findet vom \emph{4.\ bis 8.\ Mai 2016} an der  \emph{Universität Konstanz}\footnote{\href{https://zapf.uni-konstanz.de/}{\url{https://zapf.uni-konstanz.de/}}} statt.

Fragen und Anregungen können gerne an den \emph{Ständigen Ausschuss der Physik-Fachschaften}\footnote{\href{mailto:stapf@googlegroups.com}{\url{stapf@googlegroups.com}}} gerichtet werden.

Alle Stellungnahmen der ZaPF und weitere Informationen sind auf \href{http://www.zapfev.de}{\url{www.zapfev.de}} zu finden.

