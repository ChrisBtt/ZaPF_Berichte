Vom 10. bis 13. November 2016 fand in Dresden die 75. Zusammenkunft
aller deutschsprachigen Physik-Fachschaften (kurz: ZaPF) statt.  Die ZaPF ist
die Bundesfachschaftentagung der Physik und versteht sich dabei als eine
grundlegende Basis zum Austausch zwischen den Physik-Fachschaften im
deutschsprachigen Raum über hochschulpolitische Themen und darüber hinaus als
Gremium der Meinungsbildung und -äußerung der Physikstudierenden. Sie tagt
einmal pro Semester an unterschiedlichen Hochschulen, wobei sie von der
Physik-Fachschaft der ausrichtenden Hochschule selbst organisiert wird. \\

Diese Aufgabe wurde im Wintersemester 2016 von der Fachschaft Physik der Technischen Universität Dresden übernommen. 
Es nahmen 267 Fachschaftler*innen aus insgesamt 56 teilnehmenden Fachschaften teil, 
die sich in über 50 Arbeitskreisen austauschten und Positionen zu verschiedenen Themen erarbeiteten.

Schwerpunkte der ZaPF in Dresden waren die Auseinandersetzung mit dem CHE-Ranking, der Umgang mit Anträgen zur Exzellenzstrategie, der Rahmenvertrag zwischen VG Wort und KMK, 
die Kritik an der Symptompflicht auf Attesten für Prüfungsunfähigkeit, die geplanten Studiengebühren in Baden-Württemberg, 
eine Empfehlung zum Engagement für eine bessere Hochschulfinanzierung, die Entwicklung eines Konzepts für einen Studienführer, 
die Weiterführung und Auswertung der Bachelor-Master-Umfrage, die Korrektur der Positionen der ZaPF zum Lehramtsstudium, die Einbindung von Ethikinhalten ins Physikstudium, 
die Kritik an Zulassungsbeschränkungen und Grenznoten, die Verbesserung der Doktorandenvernetzung und der Umgang mit Taschenrechnern in der Schule.

\newpage

\section*{CHE}
Die ZaPF steht seit mehreren Jahren in Kontakt mit dem CHE-Ranking, bei dem sie eine beratende Funktion einnimmt. Um diese Arbeit fortzuführen, wurden in Dresden 
zwei Arbeitskreise und ein Info-Workshop zum Thema veranstaltet. In einer Diskussion zu Rankings im Allgemeinen wurden verschiedene Meinungen erläutert. 
Außerdem wurde der neue Fragebogen von CHE durchgelesen und kommentiert. Die Verbesserungsvorschläge wurden anschließend an das CHE versendet.

\section*{Exzellenzstrategie}
Anknüpfend an die Resolution zur Exzellenzinitiative aus Konstanz\footnote{\href{https://zapfev.de/resolutionen/sose16/ExIni/Exzellenzinitiative.pdf}{\url{www.zapfev.de/resolutionen/sose16/ExIni/Exzellenzinitiative.pdf}}} 
wurde die neue Exzellenzstrategie des Bundes und der Länder diskutiert. Es wurde insbesondere kritisiert, dass an vielen Universitäten mögliche Anträge zur 
Exzellenzstrategie nicht in hochschulpolitischen Gremien diskutiert werden. Da die universitäre Profilbildung erklärtes Ziel der Exzellenzstrategie ist, 
haben die Anträge möglicherweise weitreichende Auswirkungen auf die Entwicklung der Universitäten und sind somit quasi Strukturentscheidungen.\\
Es wurde angeregt, dass die Fachschaften an ihren Universitäten darauf hinwirken sollen, dass Anträge zur Exzellenzstrategie unter Einbeziehung 
aller Statusgruppen in hochschulinternen Gremien besprochen werden sollen.

\section*{VG Wort}
Der StAPF hat sich nach einem Beschluss des Plenums auf der ZaPF in Konstanz (Sommersemester 2016) mit einem offenen Brief der Fachschaftentagung Maschinenbau beschäftigt und sich dazu entschlossen, 
diesen ebenfalls als Vertretung der ZaPF zu unterstützen. Da die Studierenden wenig Informationen darüber hatten, was der Rahmenvertrag zwischen der VG Wort und den 
Ländern mit sich bringt, wurde auf der ZaPF in Dresden in einem Arbeitskreis zuerst über die Konsequenzen der Einzelabrechnung und der aktuelle Stand der Verhandlungen 
berichtet. Anschließend wurde ein weiterer offener Brief geschrieben, welcher unterstreichen sollte, dass wir, als Vertretung der Physik Studierenden, gegen die 
Einzelabrechnung sind, da diese zu einem deutlichen Mehraufwand, höheren Kosten, als auch eine Verminderung des E-Learning Angebotes. Weitere Informationen zum 
Standpunkt der ZaPF zum Rahmenvertrag finden sie in besagtem offenen 
Brief\footnote{\href{https://zapfev.de/resolutionen/wise16/VG_Wort/VG_Wort.pdf}{\url{www.zapfev.de/resolutionen/wise16/VG_Wort/VG_Wort.pdf}}}. 

\section*{Symptompflicht auf Attesten}
Wie schon auf der letzten ZaPF, hat sich in Dresden wieder ein Arbeitskreis mit der verpflichtenden Angabe von Symptomen auf ärztlichen Attesten für 
Prüfungsunfähigkeitsbescheinigungen beschäftigt. Dieses Verfahren, was an vielen Universitäten in Deutschland von den Prüfungsausschüssen angewendet wird, 
halten wir aus Gründen des Datenschutzes und der Verletzung der Privatssphäre der Studierenden für nicht tragbar. Aufbauend auf der in Konstanz zu diesem 
Thema geleisteten Arbeit wurde eine Resolution\footnote{\href{https://zapfev.de/resolutionen/wise16/Symptompflicht/symptompflicht.pdf}{\url{www.zapfev.de/resolutionen/wise16/Symptompflicht/symptompflicht.pdf}}} 
formuliert und verabschiedet, die den Gesetzgeber dazu auffordert, eine Arbeitsunfähigkeitsbescheinigung einer ärztlichen Prüfungsunfähigketisbescheinigung gleichzusetzen.

\section*{Studiengebühren in Baden-Württemberg}
Im Rahmen des Haushaltes des Landes Baden-Württemberg für das Jahr 2017 soll das Ministerium für Wissenschaft, Forschung und Kunst (MWK) 48 Mio. Euro einsparen. 
Um dieses Haushaltsloch zu stopfen, schlägt das MWK die Einführung von Studiengebühren für einzelne Studenten vor. Die ZaPF hat zu diesem Thema eine Resolution verabschiedet, 
welche die Position der Landesstudierendenvertretung Baden-Württemberg unterstützt 
\footnote{\href{https://zapfev.de/resolutionen/wise16/Reso_Studiengebuehren/Reso_Studiengebuehren.pdf}{\url{www.zapfev.de/resolutionen/wise16/Reso_Studiengebuehren/Reso_Studiengebuehren.pdf}}}. 
Aus diesem Anlass heraus, spricht sich die ZaPF auch gegen Studiengebühren jeder Form aus
\footnote{\href{https://zapfev.de/resolutionen/wise16/Positionspapier_Studiengebuehren/Positionspapier_Studiengebuehren.pdf}{\url{www.zapfev.de/resolutionen/wise16/Positionspapier_Studiengebuehren/Positionspapier_Studiengebuehren.pdf}}}.

\section*{Empfehlung zum Engagement für eine bessere Hochschulfinanzierung}
Die Unterfinanzierung von Hochschulen stellt aus Sicht der Studierenden und den Hochschulen selbst ein großes Problem dar. Um verantwortlichen Ministerien und Abgeordneten 
diesen Missstand in das Bewusstsein zu rufen, hat die ZaPF einen Handlungskatalog aufgestellt
\footnote{\href{https://zapfev.de/resolutionen/wise16/Empfehlung_zum_Engagement_fuer_eine_bessere_Hochschulfinanzierung/Hochschulfinanzierung.pdf}{\url{www.zapfev.de/resolutionen/wise16/Empfehlung_zum_Engagement_fuer_eine_bessere_Hochschulfinanzierung/Hochschulfinanzierung.pdf}}}. 
Fachschaften und Studierendenvertretungen sollen durch die Punkte dieses Kataloges auf die Problematik aufmerksam gemacht werden. Auch die Umsetzung durch andere Stellen ist 
hierbei explizit erwünscht.

\section*{Studienführer}
Die ZaPF ist seit einiger Zeit dabei einen komplett neuen Studienführer, in sehr viel größerem Ausmaß als bisher, zu erarbeiten, der Studieninteressierten 
dabei helfen soll, sich für einen Studienort zu entscheiden. Zudem soll eine Wechseldatenbank enthalten sein, die es Bachelor-Absolventen eine zum Master 
passende Universität zu finden. Die Datenbank soll von den jeweiligen Fachschaften verwaltet werden, ein Vergleich soll anhand objektiver Kriterien ermöglicht werden. 
Bei Interesse soll der Studienführer um andere Fachbereiche erweitert werden. Dafür hat sich die ZaPF in einer 
Resolution\footnote{\href{https://zapfev.de/resolutionen/wise16/Studienfuehrer/studienfuehrer.pdf}{\url{www.zapfev.de/resolutionen/wise16/Studienfuehrer/studienfuehrer.pdf}}} 
an die anderen Bundesfachschaftentagungen gewandt. Der Release des Studienführers ist aktuell für das Jahr 2019 geplant.

\section*{Bachelor-Master-Umfrage}
Im Jahr 2014 wurde eine Selbstbefragung der Physikstudierenden an den Hochschulen durchgeführt. Hauptschwerpunkte waren Fragestellungen zur Studieneingangsphase, 
Arbeitsbelastung, Übergang von Schule zu Studium und Perspektivengedanken. Da diese Umfrage Mängel aufwies, wurde die Ausarbeitung und Durchführung einer erneuten 
Umfrage zum Bachelor- und Masterstudium unter Leitung des Kommunikationsgremiums beschlossen. Die Ergebnisse der Umfrage sollen zur Sommer-ZaPF 2018 in 
Heidelberg veröffentlicht werden.

\section*{Lehramt}
Ergänzend zur Resolution der ZaPF bezüglich der Ausgestaltung des Lehramtstudiums aus 
Frankfurt\footnote{\href{https://zapfev.de/resolutionen/sose10/Lehramtstellungnahme.pdf}{\url{www.zapfev.de/resolutionen/sose10/Lehramtstellungnahme.pdf}}} 
und der Erweiterung dieser Resolution durch eine Stellungnahme in 
Wien\footnote{\href{https://zapfev.de/resolutionen/wise13/Reso_WiSe13_Fachdidaktikprofessuren.pdf}{\url{www.zapfev.de/resolutionen/wise13/Reso_WiSe13_Fachdidaktikprofessuren.pdf}}}, 
wurde auf dieser Tagung ein weiterer Nachtrag zur Resolution\footnote{\href{https://zapfev.de/resolutionen/wise16/Lehramt/Lehramt.pdf}{\url{www.zapfev.de/resolutionen/wise16/Lehramt/Lehramt.pdf}}} 
getätigt. In diesem korrigiert die ZaPF ihre Ansprüche an Fachdidaktikprofessuren. 

\section*{Ethikinhalte im Physikstudium}
Da die ZaPF Ethikinhalte während des Physikstudiums als wichtig erachtet, wurde ein Positionspapier beschlossen, welches empfiehlt, Ethikinhalte in das 
Physikstudium einzubinden\footnote{\href{https://zapfev.de/resolutionen/wise16/Ethikinhalte_im_Physikstudium/ethikinhalte.pdf}{\url{www.zapfev.de/resolutionen/wise16/Ethikinhalte_im_Physikstudium/ethikinhalte.pdf}}}.

\section*{Zulassungsbeschränkungen und Grenznoten}
Nachdem  die ZaPF auf ihrer letzten Tagung in Konstanz ein Positionspapier\footnote{\href{https://zapfev.de/resolutionen/sose16/Zulassungen/zulassungsbeschraenkungen.pdf}{\url{www.zapfev.de/resolutionen/sose16/Zulassungen/zulassungsbeschraenkungen.pdf}}} 
gegen Zulassungsbeschränkungen und Grenznoten beschlossen hat, wurde auf dieser Tagung eine Resolution\footnote{\href{https://zapfev.de/resolutionen/wise16/Zugangs-Zulassungsbeschraenkung/Reso\%20gegen\%20Zugangs-\%20und\%20Zulassungsbeschraenkungen.pdf}{\url{www.zapfev.de/resolutionen/wise16/Zugangs-Zulassungsbeschraenkung/Reso\%20gegen\%20Zugangs-\%20und\%20Zulassungsbeschraenkungen.pdf}}} 
gegen bestehende Zulassungsbeschränkungen verabschiedet. Insbesondere Grenznoten sind in jedem Fall eine falsche Handhabe von existierenden  Problemen, welche auf 
andere Weise gelöst werden sollten. Die ZaPF spricht sich dafür aus, dass das Studium und somit die Bildung ein Recht bleiben sollte und nicht anhand von 
Beschränkungen zu einem Privileg werden sollte.
		
\section*{Doktorandenvernetzung}
Bezüglich der Doktorandenvernetzung auf der ZaPF wurde zunächst über die Aktivitäten der Helmholtz juniors zur Verbesserung der Situation Promovierender informiert. 
Ein wichtiger Punkt hierbei ist die Forderung von tariflich geregelten Promotionsrahmenverträgen und  Betreuungsvereinbarungen. \\
Im Zuge des Meinungsaustausches zur weiteren Strategie der ZaPF bezüglich Doktorandenvernetzungen wurde deutlich, dass zunächst einmal eruiert werden muss, wie der 
Status Quo an den Hochschulen ist und welche Meinungen es in den Fachschaften zu diesem Thema gibt. Dazu wurde ein kurzer Fragenkatalog erarbeitet, der von den 
Fachschaften bis zur Sommer-ZaPF Berlin 2017 schriftlich beantwortet werden soll. 
		
\section*{Taschenrechner in der Schule}
Die Konferenz der deutschsprachigen Mathematikfachschaften (KoMa) hat auf ihrer Tagung 2015 in Ilmenau eine Resolution zur Verwendung von Taschenrechnern in der Schule 
verfasst. In dieser spricht sich die KoMa gegen die Verwendung von Taschenrechnern in Schulen aus. Die Begründung für diese Resolution umfasst mehrere Punkte, 
welche unterschiedliche Bereiche des Lernens, der Vergleichbarkeit von Abituren und den sozialen Umstände der Schüler. Die ZaPF hat sich dieser Resolution 
angeschlossen und durch einen weiteren, eigenen Punkt ergänzt\footnote{\href{https://zapfev.de/resolutionen/wise16/Taschenrechner/taschenrechner.pdf}{\url{www.zapfev.de/resolutionen/wise16/Taschenrechner/taschenrechner.pdf}}}. 
		
\section*{Weitere Themen}
In einer kurzen Resolution\footnote{\href{https://zapfev.de/resolutionen/wise16/Solidaritaetsbekundung/solidaritaetsbekundung.pdf}{\url{www.zapfev.de/resolutionen/wise16/Solidaritaetsbekundung/solidaritaetsbekundung.pdf}}} 
hat sich die ZaPF aus aktuellem Anlass gegen die Repression von Wissenschaftler*innen und die Einschränkung von Wissenschafts- und Reisefreiheit in der Türkei ausgesprochen 
und fordert die Bundesregierung auf in dieser Sache aktiv zu werden.\\ \\
In einem Folge-Arbeitskreis zur Zivilklausel wurde anhand der Arbeit aus Konstanz eine Resolution diskutiert und ausgearbeitet. Nach längerer Diskussion über den 
Inhalt und die Formulierungen der Resolution im Endplenum wurde sie abgelehnt, da sich eindeutig nicht auf eine gemeinsame Linie geeinigt werden konnte.
				
		
Eine Liste aller Arbeitskreise und deren Ergebnisse sind im
Reader\footnote{\href{http://www.zapfev.de/reader/Name_des_readers.pdf}{\url{www.zapfev.de/reader/Name_des_readers.pdf}}}
zur ZaPF veröffentlicht.
		
\vfill
		
Die nächste ZaPF findet vom \emph{24.\ bis 28.\ Mai 2017} an der  \emph{HU Berlin}\footnote{\href{zapf.in-berlin.de}{\url{zapf.in-berlin.de}}} statt.
		
Fragen und Anregungen können gerne an den \emph{Ständigen Ausschuss der Physik-Fachschaften}\footnote{\href{mailto:stapf@zapf.in}{\url{stapf@zapf.in}}} gerichtet werden.
		
Alle Stellungnahmen der ZaPF und weitere Informationen sind auf \href{http://www.zapfev.de}{\url{www.zapfev.de}} zu finden.
		
		
		
		
