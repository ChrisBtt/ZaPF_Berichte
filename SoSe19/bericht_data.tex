Vom 07. bis 11. Juni 2019 fand in Bonn die 80. Zusammenkunft aller deutschsprachigen Physik-Fachschaften (kurz: ZaPF) statt.
Die ZaPF ist die Bundesfachschaftentagung der Physik und versteht sich dabei als eine grundlegende Basis zum Austausch zwischen den Physikfachschaften im deutschsprachigen Raum über hochschulpolitische Themen. Darüber hinaus dient sie als Gremium der Meinungsbildung und -äußerung der Physikstudierenden.

Sie tagt einmal pro Semester an unterschiedlichen Hochschulen, wobei sie von der Physikfachschaft der ausrichtenden Hochschule selbst organisiert wird. 
Im Sommer 2019 wurde die ZaPF von der Fachschaft Physik/Astronomie der Universität Bonn geplant und durchgeführt.  
Es nahmen ca. 190 Fachschaftler*innen aus insgesamt 47 Fachschaften teil.
Diese tauschten sich in über 40 Arbeitskreisen aus und erarbeiteten Positionen zu verschiedenen Themen.

Schwerpunkte der ZaPF in Bonn waren die Auseinandersetzung mit der Unterschriftenkampagne zur Zivilklausel aus Nordrhein-Westfalen, 
der BAfÖG-Novellierung, Akkreditierungsrichtlinien, Lernzielen und Rahmenbedingungen für physikalische Fortgeschrittenenpraktika, aktuelle Entwicklungen im Wissenschaftssystem, die Vermittlung von Programmierkenntnissen im Studium und Wissenschaftskommunikation.

\section*{BAföG}

Der Entwurf der Bundesregierung zur BAföG-Novellierung wurde diskutiert.
Diese Änderungen verbessern den Status Quo, sind allerdings immer noch nicht zufriedenstellend (siehe hierzu auch die Resolution vom November 2018 \footnote{\url{https://zapfev.de/resolutionen/wise18/Reso\_BAf\%C3\%B6G/BAf\%C3\%B6G.pdf}}).
So kompensiert die Novellierung nur die Kostenerhöhungen der vergangenen Jahre, auf zukünftige Entwicklungen kann damit aber nicht flexibel reagiert werden. 
Zudem wird die Förderungshöchstdauer nach wie vor an der Regelstudienzeit bestimmt, die nur von einer Minderheit von Studierenden eingehalten wird. Daher sollte die Förderungshöchstdauer um mindestens ein Semester verlängert werden.
Das BAföG Bündnis hat hierzu ein Positionspapier \footnote{\url{https://www.bmbf.de/files/Stellungnahme\%20BAf\%c3\%b6G-B\%c3\%bcndnis.pdf}} verfasst, welchen die ZaPF unterstützt.
		
\section*{Akkreditierungsrichtlinien der ZaPF (kommentierte MRVO)}

2018 haben sich im deutschen Akkreditierungssystem Änderungen ergeben, zu denen die ZaPF sich bereits geäußert hat. 
Auf der diesjährigen ZaPF wurde nun nach Vorbereitungen im letzten Jahr die Musterrechtsverordnung (MRVO) kommentiert, 
sodass die Akkreditierungsrichtlinien der ZaPF Berücksichtigung finden, um somit den von der ZaPF entsendeten Gutachter*innen die Arbeit zu erleichtern.

Zusätzlich soll eine kondensierte Version als Empfehlung für die Akkreditierungsagentur ASIIN entworfen werden, da diese ihre Richtlinien für die Akkreditierung von Physikstudiengängen gerade überarbeitet. 

\section*{Beschäftigung mit aktuellen Entwicklungen im Wissenschaftssystem}

Die ZaPF hat sich mit aktuellen Entwicklungen im Wissenschaftssystem auseinandergesetzt, darunter der Einführung der Nationalen Forschungsdateninfrastruktur (NFDI \footnote{\url{https://www.dfg.de/foerderung/programme/nfdi/}}) und Themen rund um Open Science. 
Zur NFDI wurde eine Referentin von der DFG eingeladen, die die Initiative vorstellte und anschließend mit den Teilnehmenden über die Ausgestaltung der NFDI und insbesondere ihre Rolle im Physikstudium diskutierte.
Schwerpunkt in der Diskussion um Open Science waren die Themen Plan S \footnote{\url{https://www.coalition-s.org/about/}}, Project DEAL \footnote{\url{https://www.projekt-deal.de/}} und wie Open Science in der universitären Lehre verankert werden könnte.

\section*{Bachelor-Master-Umfrage}

Im Sommersemester 2018 wurde durch die ZaPF eine Umfrage unter den Physikstudierenden im deutschsprachigen Raum durchgeführt, welche zum Ziel hat, die Entwicklung des Physikstudiums in regelmäßigen Abständen zu beobachten. 
Die Umfrage wurde zum dritten Mal durchgeführt und es nahmen 2750 Studierende aus dem Bachelor, Master oder Promotion daran teil. 
Es wurden hierbei allgemeine Fragen zu Bereichen wie Studienbelastung, subjektiver Wahrnehmung der Vermittlung von Qualifikationen und Kompetenzen, Vorbereitung auf das Studium sowie der Weg nach dem Studium gestellt. 
Darüber hinaus wurde analysiert, inwieweit die Themenkomplexe Ethik und Wissenschaftstheorie im Studium eingebunden werden sollten.

Auf der ZaPF in Bonn hat sich ein Arbeitskreis darüber ausgetauscht, wie die Ergebnisse der Umfrage den Universitäten aufbereitet zur Verfügung gestellt werden können und an welchen Stellen sie außerdem eingesetzt werden können.
Es wird an einer öffentlich zugänglichen Website gearbeitet.
Außerdem ist eine Versendung der hochschulspezifischen Resultate an die entsprechenden Fachschaften geplant.

\section*{Qualifikationsziele physikalischer Praktika}

Praktika sind eine zentrale Lehrveranstaltungsform in naturwissenschaftlichen Fächern. Daher wurde sich bereits auf vergangenen ZaPFen mit 
Qualifikationszielen und Rahmenbedingungen physikalischer Anfangs- und Fortgeschrittenenpraktika beschäftigt. 
Die Sommer-ZaPF 2019 hat ein Positionspapier zur Verbesserung von Fortgeschrittenenpraktika in der Physik verabschiedet \footnote{\url{https://zapfev.de/resolutionen/sose19/Fortgeschrittenenpraktika/Fortgeschrittenenpraktika.pdf}}. 
Das Ziel dieser Fortgeschrittenenpraktika ist die Vermittlung von spezifischen inhaltlichen sowie formellen Qualifikationszielen und Schlüsselqualifikationen. 
Diese Ziele sind zentrale Fähigkeiten Studierender der Physik und sollen dabei im Fortgeschrittenenpraktikum erlernt und ausgebaut werden. 
Nach erfolgreichem Abschluss sollen Qualifikationen wie vertiefte Statistik- und Plotkenntnisse, ein nachhaltiges Forschungsdatenmanagement sowie ein übersichtliches Dokumentieren und Auswerten der gemessenen Daten vermittelt worden sein. 
Übergeordnetes Ziel ist dabei, eine Intuition für physikalische Zusammenhänge zu entwickeln.
Die Gestaltung und Vermittlung dieser Qualifikationsziele obliegt dabei der Hochschule. 

\section*{Programmierkentnisse}

Die Vermittlung von Programmierkenntnissen ist ein vitaler Teil des Physikstudiums. 
Seit einiger Zeit beschäftigt sich die ZaPF daher damit, wie diese idealerweise im Studium integriert sein sollte. 
Auf der letzten ZaPF in Bonn wurde deswegen hierzu ein Entwurf eines Leitfadens erstellt. Dieser wird voraussichtlich auf der kommenden ZaPF in Freiburg verabschiedet werden. 
Hierbei werden Präsenzprogrammierübungen im Gegensatz zu reinem Frontalunterricht als geeignete Methode zur Vermittlung von solchen Fähigkeiten vorgeschlagen.

\section*{Wissenschaftskommunikation}

Seit dem Wintersemester 2016 beschäftigt sich die ZaPF mit dem Thema  Wissenschaftskommunikation (siehe Postionspapiere von 2018 \footnote{\url{https://zapfev.de/resolutionen/wise18/PosPap_WissKomm_I/WissKomm_I.pdf}} \footnote{\url{https://zapfev.de/resolutionen/wise18/PosPap_WissKomm_II/WissKomm_II.pdf}}.
In diesem Semester wurden die erarbeiteten Forderungen und Umsetzungshinweise  zusammengefasst, die auf eigenen Erfahrungen, Gesprächen mit Wissenschaftskommunikator*innen und Schulungen basieren.
Außerdem wurde sich ausgetauscht, wie man sich in Diskussionen zu Pseudowissenschaften und falschen Darstellungen am besten äußert.

\section*{Nachhaltigkeit und Diversität}

Ein weiteres Thema war die Nachhaltigkeit an Hochschulen, in der Fachschaftsarbeit und dem eigenen Handeln.
Aus dem Austausch entstand eine Liste mit Handlungsvorschlägen. Diese enthält unter anderem die Einrichtung von Beauftragen für Umweltthemen in den einzelnen Fachschaften.
Außerdem werden Arbeitskreise auf den kommenden ZaPFen folgen, um ein nachhaltiges Handeln langfristig sowohl in der Fachschaftsarbeit als auch auf der ZaPF und im Privatleben zu etablieren.
Auch über den Umgang mit Geschlechterdiversität und gendergerechte Sprache wurde beraten.

\section*{Allgemeine Hochschulpolitik}

Neben Themen mit direktem Bezug zum Physikstudium hat die ZaPF sich auch Themen der allgemeineren Hochschulpolitik gewidmet. 
Hierzu zählen Positionierungen gegen Denunziationsplattformen für Lehrpersonen\footnote{\url{https://zapfev.de/resolutionen/sose19/Lehrerpranger/Lehrerpranger.pdf}},
für eine demokratische Hochschule
\footnote{\url{https://zapfev.de/resolutionen/sose19/Reso_Selbstverwaltung/Reso_Selbstverwaltung.pdf}}
oder eine Erneuerung der Position, dass Verfasste Studierendenschaften zur Vertretung der Rechte und Interessen der Studierenden notwendig sind\footnote{\url{https://zapfev.de/resolutionen/sose19/Reso_VS/Reso_VS.pdf}}.
Außerdem unterstütze die ZaPF die Unterschriftenkampagne "Wissenschaft für Nachhaltigkeit, Frieden
und Demokratie – Die Zivilklausel in NRW erhalten!"
\footnote{\url{https://zapfev.de/resolutionen/sose19/Unterschriftenkampagne_Zivilklausel/Reso_Unterschriftenkampagne.pdf}}.

\section*{Sonstiges}
Neben den genannten Themen hat sich die Sommer-ZaPF 2019 mit zahlreichen weiteren Themen befasst, hier jedoch bislang noch keine abschließende Position ausformuliert. Diese Themen werden auch die kommenden ZaPFen weiter begleiten. 
Eine Fragestellung waren etwa Konzepte für Bibliotheken und Lernräume.
Hierbei wurde sich ausgetauscht, welche Konzepte an den verschiedenen Universitäten bisher umgesetzt werden. 
Weiterhin wurden Ideen gesammelt, wie sowohl das Lernen für Studierende so angenehm und ablenkungsfrei wie möglich gestaltet werden kann als auch die finanziellen Mittel der entsprechenden Universitäten nicht zu überlasten. 
Auch die Arbeit an einer Positionierung zu Bearbeitungszeiten von Abschlussarbeiten wurde fortgesetzt.
Hierbei besteht das grundsätzliche Problem darin, die Einhaltung rechtlicher Rahmenbedingungen sicherzustellen. Auf einer zukünftigen ZaPF soll hierzu eine gemeinsame Position gefasst werden.
Außerdem wurde sich darüber ausgetauscht, inwieweit die Umsetzung einer Tagung der Physikfachschaften beispielsweise auf europärischer Ebene möglich ist.
Hierzu sollen für die nächste ZaPF in Freiburg benachbarte Fachschaften aus Frankreich eingeladen werden. 

		
\vfill
		
Die nächste ZaPF findet vom \emph{31.\ Oktober bis 3.\ November 2019} an der  \emph{Albert-Ludwigs-Universität} in Freiburg \footnote{\url{https://www.fachschaft.physik.uni-freiburg.de/zapf/}} statt.
		
Fragen und Anregungen können gerne an den \emph{Ständigen Ausschuss der Physik-Fachschaften}\footnote{\href{mailto:stapf@zapf.in}{\url{stapf@zapf.in}}} gerichtet werden.
		
Alle Stellungnahmen der ZaPF und weitere Informationen sind auf \href{http://www.zapfev.de}{\url{www.zapfev.de}} zu finden.
		
		
		
		
