\documentclass[a4paper]{article}
\usepackage[T1]{fontenc}
\usepackage[utf8]{inputenc}
\usepackage{lmodern}
\usepackage{footnote}
\usepackage{url}
\usepackage{geometry}
\usepackage{hyperref}
\usepackage[english,ngerman]{babel}
\geometry{
left=3.0cm,
right=3.0cm,
top=3.0cm,
bottom=3.0cm}
\begin{document}
	
\title{Bericht des StAPF zur 76. Zusammenkunft aller Physikfachschaften}
\date{20. Juli 2017}

\maketitle

Vom 24. bis 28. Mai 2017 fand in Berlin die 76. Zusammenkunft
aller deutschsprachigen Physik-Fachschaften (kurz: ZaPF) statt.  Die ZaPF ist
die Bundesfachschaftentagung der Physik und versteht sich dabei als eine
grundlegende Basis zum Austausch zwischen den Physik-Fachschaften im
deutschsprachigen Raum über hochschulpolitische Themen und darüber hinaus als
Gremium der Meinungsbildung und -äußerung der Physikstudierenden. Sie tagt
einmal pro Semester an unterschiedlichen Hochschulen, wobei sie von der
Physik-Fachschaft der ausrichtenden Hochschule selbst organisiert wird. 
\\

Diese Aufgabe wurde im Sommersemester 2017 von der Fachschaft Physik der Technischen Universität Berlin, der Freien Universität Berlin, der Humboldt Universität Berlin und der Universität Potsdam übernommen. 
Es nahmen 198 Fachschaftler*innen aus insgesamt 56 teilnehmenden Fachschaften teil, 
die sich in über 40 Arbeitskreisen austauschten und Positionen zu verschiedenen Themen erarbeiteten.

Schwerpunkte der ZaPF in Berlin waren die Auseinandersetzung mit dem CHE-Ranking, der Umgang mit Anträgen zur Exzellenzstrategie, die aktuellen Verhandlungen zwischen VG Wort und KMK, die eingetretenen Studiengebühren in Baden-Württemberg, die Ablehnung des studentischen Vertreters im Akkreditierungsrat durch die HRK, 
die Weiterführung der Bachelor-Master-Umfrage und das Studieneingangswissen von Abiturienten.

\newpage

\section*{VG-WORT}
Anknüpfend an die Arbeitskreise vergangener ZaPFen wurde die aktuelle Situation zur Abrechnung der Nutzung von Schriftwerken durch die VG WORT und dem Urheberrechtsgesetz evaluiert. 

In Zusammenarbeit mit der Fachschaftentagung Maschinenbau (FaTaMa) im deutschsprachigen Raum wurde ein offener Brief \footnote{\href{https://zapfev.de/resolutionen/sose17/vg_wort/vg_wort.pdf}{\url{https://zapfev.de/resolutionen/sose17/vg_wort/vg_wort.pdf}}} auf Grundlage eines Briefes der Konferenz aller Werkstofftechnischen und Materialwissenschaftlichen Studiengänge (KaWuM) verfasst. Er begrüßt die aktuellen Entwicklungen zur Änderungen des Uhrheberrechtsgesetzes, warnt aber davor, dass Konzepte der digitalen Lehre wie Blended Learning, digitale Hochschulen, etc. in der vorliegenden Form nicht berücksichtig werden. Weiterhin wird gefordert, dass die derzeitige Verhandlungsphase um die Abrechnung von Schriftwerken zwischen Kultusministerkonferenz, Hochschulrektorenkonferenz und VG WORT bis zum Inkrafttreten des Gesetzes verlängert wird. Zudem sollen Vertreter der Studierendenschaften als Teil der Abordnung der Hochschulen in die Verhandlungen mit einbezogen werden.


\section*{Akkreditierungsrat}
Die Hochschulrektorenkonferenz (HRK) lehnt den vom Poolvernetzungstreffen (PVT) entsandten studentischen Kandidaten für den Akkreditierungsrat ohne Begründung ab und bestimmt einen eigenen Kanditaten ohne Rücksprache mit dem PVT. Dieser ist nicht von der Statusgruppe der Studierenden demokratisch legitimiert. Mehrere Versuche einer Kontaktaufnahme seitens des studentischen Akkreditierungspools blieben ohne Erfolg.\\
Daraufhin hat die ZaPF sich auf ihrer Tagung gemeinsam mit dem KASAP (Koordinierungsausschuss des studentischen Akkreditierungspools) in einer Resolution\footnote{\href{https://zapfev.de/resolutionen/sose17/Akkreditiertungsrat/Reso\%20Akkreditierungsrat.pdf} {\url{https://zapfev.de/resolutionen/sose17/Akkreditiertungsrat/Reso\%20Akkreditierungsrat.pdf}}} zu diesen Vorfällen positioniert und fordert die HRK dazu auf, die von Studierenden selbst auf demokratischem Wege bestimmten studentischen VertreterInnen zu benennen und zu einer konstruktiven und kommunikativen Zusammenarbeit zurückzukehren. Desweiteren wurde im Juni eine Pressemitteilung\footnote{\href{https://zapfev.de/resolutionen/sose17/Akkreditiertungsrat/pressemitteilung_ZaPF.pdf} {\url{https://zapfev.de/resolutionen/sose17/Akkreditiertungsrat/pressemitteilung\_ZaPF.pdf}}} mit dieser Thematik herausgegeben. 


\section*{Bauvorhaben}
Bauvorhaben an Universitäten und Hochschulen werden meist über sehr große Zeiträume geplant. Bei dieser Planung werden die Studierenden als größte Nutzungsgruppe jedoch meist außer Acht gelassen.
Eine gemeinsame Planung von Neu- und Umbauten kann zu einer besseren Auslastung aller Räumlichkeiten als Aufenthalts-, Arbeits- und Erholungsräume führen und somit die Situation für alle Studierenden verbessern. Vor diesem Hintergrund hat die ZaPF in Berlin eine Resolution beschlossen, welche fordert, dass die Studierenden einen festen Bestandteil in Planungskommissionen bilden\footnote{\href{https://zapfev.de/resolutionen/sose17/Bauprojekte/reso_bauprojekte.pdf}{\url{https://zapfev.de/resolutionen/sose17/Bauprojekte/reso_bauprojekte.pdf}}}.

\section*{Grund- und Anfängerpraktika}
Grund- bzw. Anfängerpraktika bilden einen essentiellen Bestandteil des Physik Stuidums. Die Anforderungen und Lernziele varriieren hierbei jedoch zwischen den einzelnen Universitäten. Nach mehreren Tagungen hat die ZaPF sich auf einen Katalog\footnote{\href{https://zapfev.de/resolutionen/sose17/Praktika/PosPapier_Praktika.pdf}{\url{https://zapfev.de/resolutionen/sose17/Praktika/PosPapier_Praktika.pdf}}} an Fähigkeiten geeinigt, welche Studierenden nach einem entsprechenden Praktikum vorweisen sollen. Dieser Katalog umfasst neben der korrekten Durchführung von Experimenten auch die Auswertung und Darstellung von Messdaten. Dies soll den Studierenden eine gute wissenschaftliche Praxis vermitteln. 
Wie diese Lernziele vermittelt werden, obliegt hierbei jeder Universität selbst.

\section*{Exzellenzstrategie}
Anknüpfend an die verschiedenen Arbeitskreise zur Exzellenzinitiative in den vergangenen Semestern  und der Resolution aus Konstanz\footnote{\href{https://zapfev.de/resolutionen/sose16/ExIni/Exzellenzinitiative.pdf}{\url{www.zapfev.de/resolutionen/sose16/ExIni/Exzellenzinitiative.pdf}}}, wurde auf der ZaPF in Berlin in einem Arbeitskreis mit Untersützung von Herrn Eiserle von der TU Berlin eine Handreichung für die Studierenden entwickelt. Mit dieser Handreichung sollen Studierende grundlegend über die Exzellenzstrategie informiert und dazu ermutigt werden, sich an den jeweiligen Prozessen an der eigenen Universität zu beteiligen. Des Weiteren fordert die ZaPF die Universitäten dazu auf, diese Partizipation der Studierenden zu ermöglichen. 

\section*{Schaffung permanenter Stellen im wissenschaftlichen Mittelbau}
Die ZaPF hat sich auf Grund des \emph{Tenure-Track-Programms} des Bundesministeriums für Bildung und Forschung mit der Problematik der wissenschaftliche Mittelbaustellen beschäftigt. Die ZaPF spricht sich für die Verbesserung der Situation durch Schaffung neuer, vor allem unbefristeter, Stellen aus, um die Qualität und Kontinuität von Lehre und langfristigen Forschungsvorhaben zu verbessern. Das aktuelle Vorhaben zur Förderung von Tenure-Track-Stellen hält die ZaPF nicht für ein geeignetes Mittel, da es sich zu sehr auf die Professur als Karriereziel fokussiert und die Schaffung einer ausgeglichenen akademischen Landschaft behindert. Daher hat die ZaPF zu dieser Thematik eine Resolution\footnote{\href{https://zapfev.de/resolutionen/sose17/mittelbau/mittelbau.pdf}{\url{www.zapfev.de/resolutionen/sose17/mittelbau/mittelbau.pdf}}} gefasst.



\section*{Symptompflicht auf Attesten}
Auf der ZaPF in Dresden wurde eine Resolution\footnote{\href{https://zapfev.de/resolutionen/wise16/Symptompflicht/symptompflicht.pdf}{\url{www.zapfev.de/resolutionen/wise16/Symptompflicht/symptompflicht.pdf}}} zu der von einigen Ländern und Hochschulen gewählten Praxis, dass auf Prüfungsunfähigkeitsbescheinigungen Krankheitssymptome verpflichtend angegeben werden müssen, verabschiedet.\\
Nach der Veröffentlichung der Resolution in Dresden hat die ZaPF dazu von verschiedenen Stellen Feedback erhalten. Dies resultierte in einem erneuten Arbeitskreis zum Thema Symptompflicht. Daraufhin wurde die in Dresden als Resolution gefasste Position nochmals durch die Verabschiedung eines Positionspapiers\footnote{\href{https://zapfev.de/resolutionen/sose17/symptompflicht/PosPapier_Symptompflicht.pdf}{\url{www.zapfev.de/resolutionen/sose17/symptompflicht/PosPapier_Symptompflicht.pdf}}} bekräftigt.


\section*{Studiengebühren}
Zum Wintersemester 2017/18 werden in Baden-Württemberg Studiengebühren für Studierende aus nicht-EU Ländern, sowie für Zweitstudierende erhoben. Des Weiteren zeigen sich in anderen Bundesländern ebenfalls Bemühungen, Studiengebühren erneut einzuführen.
Nach der Aussprache der ZaPF in Dresden gegen die Studiengebühren in Baden-Württemberg und Gebühren im Allgemeinen in einer Resolution \footnote{\href{https://zapfev.de/resolutionen/wise16/Reso_Studiengebuehren/Reso_Studiengebuehren.pdf}{\url{https://zapfev.de/resolutionen/wise16/Reso_Studiengebuehren/Reso_Studiengebuehren.pdf}}}, wurde auf der ZaPF in Berlin die Haltung gegen Studiengebühren jeglicher Art in einer Resolution\footnote{\href{https://zapfev.de/resolutionen/sose17/studiengebuehren/studiengebuehren.pdf}{\url{https://zapfev.de/resolutionen/sose17/studiengebuehren/studiengebuehren.pdf}}} gefestigt.

        
\section*{Abiturwissen}
Bezugnehmend auf Resolutionen und Stellungnahmen der Deutschen Mathematiker-Vereinigung, der Gesellschaft für Didaktik der Mathematik und des Verbands zur Förderung des MINT-Unterrichts zur Qualität der Kompetenzstandards und zum daraus resultierenden Mathematikkenntnisstand der Studienanfänger*innen hat die ZaPF sich eingehend mit der Bedeutung des Mathematikunterrichts für das Physikstudium sowie mit den Anforderungen des Mathematikabiturs befasst. In diesem Zusammenhang wurde auch über die Entwicklung und den Stand der Mathematikvorkurse an den Universitäten sowie über den Einsatz verschieden komplexer Taschenrechner in der Schule gesprochen. In dieser Sache wurde ein Positionspapier\footnote{\href{https://zapfev.de/resolutionen/sose17/abiwissen/PosPapier_abiwissen.pdf}{\url{www.zapfev.de/resolutionen/sose17/abiwissen/PosPapier_abiwissen.pdf}}}
verfasst, in dem die ZaPF sich klar für einen studienvorbereitenden und inhaltlich anspruchsvolleren Unterricht durch entsprechende Kompetentzstandards und Kerncurricula ausspricht.


\section*{Gesellschaftliche Verantwortung und Zivilklausel} 
Das Thema Zivilklausel wurde schon auf verschiedenen ZaPFen diskutiert. 
Dabei besteht grundsätzlich Einigkeit über die gesellschafltiche Verantwortung der Hochschulen, Rüstung, Kriegsvorbereitung oder -durchführung nicht zu unterstützen, jedoch gibt es zu verschiedenen Formulierungen immer wieder intensive Diskussionen. Insbesondere wurden einzelne Formulierungen von Hochschulen und Landeshochschulgesetzen besprochen.\\
 In Berlin wurde nun ein Positionspapier\footnote{\href{https://zapfev.de/resolutionen/sose17/gesellschaftlich_verantwortung/PosPapier_gesellschaftliche_verwantwortung.pdf} {\url{www.zapfev.de/resolutionen/sose17/gesellschaftlich_verantwortung/PosPapier_gesellschaftliche_verwantwortung.pdf}}} 
verabschiedet, in dem sich die ZaPF klar dafür ausspricht, \glqq dass die Hochschulen einen Beitrag zu einer nachhaltigen, friedlichen und demokratischen Welt \grqq~leisten. In diesem Zusammenhang betont das Positionspapier auch die Notwendigkeit der Unabhängigkeit der Universitäten und universitären Lehre und Forschung.


\section*{CHE-Ranking} 
Die  ZaPF steht seit mehreren Jahren in Kontakt mit dem CHE-Ranking, bei dem  sie eine beratende Funktion einnimmt. Um diese Arbeit fortzuführen, wurde in Berlin ein Arbeitskreis zu  diesem Thema veranstaltet. Der anstehende Fragebogen vom CHE wurde durchgearbeitet und kommentiert, da er sich für die Physik stark verändert hat. Wie sich diese Änderung auf das Ergebnis und die Aussagekraft der Umfrageergebnisse auswirken wird, werden wir weiter verfolgen.


\section*{Ethikinhalte im Physikstudium}
Auf der ZaPF in Dresden wurde ein Positionspapier\footnote{\href{https://zapfev.de/resolutionen/wise16/Ethikinhalte_im_Physikstudium/ethikinhalte.pdf}{\url{www.zapfev.de/resolutionen/wise16/Ethikinhalte_im_Physikstudium/ethikinhalte.pdf}}}  verabschiedet, in dem sich die ZaPF dafür ausspricht, Ethikinhalte in das Physikstudium einzubinden. 
Um  mehr Information über den aktuellen Stand an den Universitäten zu  erhalten, fand nun auf der ZaPF in Berlin ein weiterer Arbeitskreis zu  diesem Thema statt. Dort ging es vor allem darum, eine Umfrage für  Physikfachschaften und Studierende zu entwerfen, die ermitteln soll, inwiefern  sich innerhalb der aktuellen Studiengänge mit der ethischen  Verantwortung der Wissenschaft auseinander gesetzt wird und was die  Studierenden darunter verstehen.


\section*{Bachelor-Master-Umfrage}
Im  Jahr 2014 wurde eine Selbstbefragung der Physikstudierenden an den  Hochschulen durchgeführt. Hauptschwerpunkte waren Fragestellungen zur  Studieneingangsphase, 
Arbeitsbelastung,  Übergang von Schule zu Studium und Perspektivengedanken. Da diese  Umfrage Mängel aufwies, wurde die Ausarbeitung und Durchführung einer  erneuten 
Umfrage zum Bachelor- und Masterstudium unter Leitung des Kommunikationsgremiums beschlossen. Der  aktuelle Stand der Arbeit zur Umfrage wurde in einem Arbeitskreis berichtet. Leider kann die Umfrage inklusive Auswertung nicht wie geplant bis zum Sommersemester 2018  durchgeführt werden, da noch einige inhaltliche Fragen geklärt werden müssen. Hierfür hat die ZaPF ein Positionspapier  veröffentlicht als Richtlinie für kommende Umfragen\footnote{\href{https://zapfev.de/resolutionen/sose17//PosPapier_BaMa_Umfrage/PosPapier_BaMa_Umfrage.pdf}{\url{https://zapfev.de/resolutionen/sose17/
/PosPapier_BaMa_Umfrage/PosPapier_BaMa_Umfrage.pdf}}}. Der neue Termin zur Durchführung der Umfrage ist das Sommersemester 2018.



        
\section*{Weitere Themen}

Die ZaPF hat sich außerdem noch mit den folgenden Themen beschäftigt:

\subsection*{BAföG}
Es wurden Aspekte des BAföG gesammelt, die als problematisch angesehen werden.
Hierzu wird es einen weiteren Arbeitskreis in Siegen geben, um diese Punkte in Verbindung mit der aktuellen Sozialerhebung\footnote{\href{http://www.sozialerhebung.de/download/21/Soz21_hauptbericht.pdf}{\url{http://www.sozialerhebung.de/download/21/Soz21_hauptbericht.pdf}}} zu diskutieren. 

\subsection*{Hochschulgesetz Thüringen}
Es wurde sich intensiv mit der Novellierung des Hochschulgesetzes in Thüringen beschäftigt. Dabei wurden viele Stellen im Gesetz mit großer Sorge gesehen. Es ist eine weitere Beschäftigung mit diesem Thema geplant.

\subsection*{Lehramt}
Der Arbeitskreis Lehramt hat auf eine Rückmeldung des Fachverbandes Didaktik der Deutschen Physikalischen Gesellschaft (DPG) reagiert und ein gemeinsames Gespräch unter Teilnahme der Vertreter der Gesellschaft für Didaktik der Chemie und Physik (GDCP) für die ZaPF in Siegen vorbereitet.

\subsection*{Physikveranstaltung für Nichtphysikstudierende}
Es wurde ein Austausch über die Situation von Physik Veranstaltungen als Nebenfach durchgeführt. Ähnlich wie zum Thema Praktikum, soll in Zukunft eine Position der ZaPF hierzu erarbeitet werden. 

\subsection*{Rote Fäden der Studienreform}
Zum Thema Motivation und Curriculum im Studium wurde diskutiert, welche verschiedenen Ansätze es gibt und welche Methoden und Curriculae gut geeignet sein könnten um gute Rahmenbedingungen für Studierende zu schaffen.

\subsection*{Studienführer}
Der Studienführer soll weiterentwickelt werden. Um zielführender an der Fertigstellung zu arbeiten, wurde begonnen, die Arbeit umzustrukturieren.

\subsection*{Vernetzung mit Doktorandenvertretung}
Das Thema Doktorandenvertretung wurde wieder fortgeführt und diskutiert. Dieses mal wurde der Arbeitskreis insbesondere als Vorbereitung auf eine durch die jDPG organisierte Podiumsdiskussion zum Thema \emph{Finanzierung des Mittelbaus und der Promotion} genutzt.

\subsection*{Wissenschaftskommunikation}
Der Grundstein für eine Positionierung der ZaPF zu Wissenschaftskommunikation von und durch Studierende wurde durch einen Austausch-AK gelegt. Weiterführende Arbeit auch mit externen Referenten ist für die Zukunft geplant.

\subsection*{Workshop Akkreditierung}
Da die ZaPF Mitglieder in den studentischen Akkreditierungpool entsendet, gibt es regelmäßig Workshops, um Interessierte an das Thema Akkreditierung heranzuführen.\\
\\
\textbf{
Eine Liste aller Arbeitskreise und deren Ergebnisse sind im
Reader\footnote{\href{http://www.zapfev.de/reader/2017_SoSe_Berlin_lang.pdf}{\url{www.zapfev.de/reader/2017_SoSe_Berlin_lang.pdf}}}
zur ZaPF veröffentlicht.}

        
\vfill
~\\       
Die nächste ZaPF findet vom \emph{28.\ Oktober bis 01.\ November 2017} an der  \emph{Uni Siegen}\footnote{\href{https://siegen.zapf.in/}{\url{siegen.zapf.in}}} statt.
\\
\\
Fragen und Anregungen können gerne an den \emph{Ständigen Ausschuss der Physik-Fachschaften}\footnote{\href{mailto:stapf@zapf.in}{\url{stapf@zapf.in}}} gerichtet werden.
\\
\\
Alle Stellungnahmen der ZaPF und weitere Informationen sind auf \footnote{\href{http://www.zapfev.de}{\url{www.zapfev.de}}} zu finden.
        
        
        



\end{document}
