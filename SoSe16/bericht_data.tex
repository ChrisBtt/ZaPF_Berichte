Vom 04. bis 08. Mai 2016 fand in Konstanz die 74. Zusammenkunft
aller deutschsprachigen Physik-Fachschaften (kurz: ZaPF) statt.  Die ZaPF ist
die Bundesfachschaftentagung der Physik und versteht sich dabei als eine
grundlegende Basis zum Austausch zwischen den Physik-Fachschaften im
deutschsprachigen Raum über hochschulpolitische Themen und darüber hinaus als
Gremium der Meinungsbildung und -äußerung der Physikstudierenden. Sie tagt
einmal pro Semester an unterschiedlichen Hochschulen, wobei sie von der
Physik-Fachschaft der ausrichtenden Hochschule selbst organisiert wird.

Diese Aufgabe wurde im Wintersemester 2015 von der Fachschaft Physik der Universtiät Konstanz übernommen. Es nahmen ??? Fachschaftler*innen aus
insgesamt ?? teilnehmenden Fachschaften teil, die sich in über ?? Arbeitskreisen
austauschten und Positionen zu verschiedenen Themen erarbeiteten.

Schwerpunkte der ZaPF in Frankfurt waren 

\newpage


\section*{Weitere Themen}

Eine Liste aller Arbeitskreise und deren Ergebnisse sind im
Reader\footnote{\href{http://www.zapfev.de/reader/}{\url{http://www.zapfev.de/reader/}}}
zur ZaPF veröffentlicht.

\vfill

Die nächste ZaPF findet vom \emph{10.\ bis 13.\ November 2016} an der  \emph{Technischen Universität Dresden}\footnote{\href{https://zapf.pfsr.de/}{\url{https://zapf.pfsr.de/}}} statt.

Fragen und Anregungen können gerne an den \emph{Ständigen Ausschuss der Physik-Fachschaften}\footnote{\href{mailto:stapf@googlegroups.com}{\url{stapf@googlegroups.com}}} gerichtet werden.

Alle Stellungnahmen der ZaPF und weitere Informationen sind auf \href{http://www.zapfev.de}{\url{www.zapfev.de}} zu finden. 
